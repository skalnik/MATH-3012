\documentclass[12pt]{article}
\usepackage{graphicx}
\usepackage{verbatim}
\usepackage{amsmath}
\title{Homework 1}
\author{Mike Skalnik}

\providecommand{\e}[1]{\ensuremath{\times 10^{#1}}}
\newcommand{\s}[1]{\setcounter{enumi}{#1}}

\begin{document}

\begin{flushright}{\large Homework 1\\ Mike Skalnik}\end{flushright}

\begin{enumerate}
  \s{1}
  
  \item
    For each digit position, there are 5 possible digits. Since there are 2n total digit positions, we can determine that there are a total of $5^{2n}$ possible positive integers.
  \item
    \begin{enumerate}
      \item Bob's password rules only allow for
        $\left( 26 + 26 \right)^3 \times 10^2 \times 10 \times 26^2 \times 10 \times 10 = 9505100800000$
        passwords
      \item The total number of stings of length 10 made from the alphabet of all uppercase \& lowercase English letters, decimal digits, and 10 symbols is
        $\left( 26 + 26 + 10 + 10 \right)^{10} = 3.7439\e{18}$
    \end{enumerate}
 
  \s{5}
  
  \item
    \begin{enumerate}
      \item Any one student from the whole group can take the first position. In the second position there are 9 possible students to choose from, and 8 for the third position. This continues until the final position, so there are simply $10! = 3628800$ possible ways for the students to line up.
      \item For the 3 positions, it must be a student from group 1, 2, and then 3. The following 3 are the same, however there is now one less student to pick from in each group, so there are only 9 to choose from for those positions. The next 3 are the same way, but only with 8. This continues until the line is complete, so we can determine that there are a total of $10!^3 = 4.7785\e{19}$ 
    \end{enumerate}
  
  \s{10}

  \item 
    \begin{enumerate}
      \item For every donut he purcahses, he can select any of the 12. This means that there are a total of $12^6 = 2985984$ possible ways of purchasing donuts.
      \item Since he wants to choose 6 distinct donuts out of 12 total, we can determine all possible ways to purchase donuts with a permutation. $P(12, 6) = 655280$
      \item In this case, the order doesn't matter at all. So, using a combination, we can determine that the total number of ways to pick donuts is $\binom{12}{6} = 924$.
    \end{enumerate}
  \item Assuming there are no positions in korfball, the selection order doesn't matter. So there are a total of $\binom{7}{4} \times \binom{11}{4} = 11550$
  \item
    \begin{enumerate}
      \item There are $P(20,4) = 116280$ possible outcomes for the competition
      \item With the honorable mention certificates, there are a total of $P(20,4) \times C(16,4) = 211629600$ possible outcomes.
    \end{enumerate}

  \s{14}
  
  \item
\end{enumerate}
\end{document}

