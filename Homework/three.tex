\documentclass[12pt]{article}
\usepackage{graphicx}
\usepackage{verbatim}
\usepackage{amsmath}
\title{Homework 3}
\author{Mike Skalnik}

\providecommand{\e}[1]{\ensuremath{\times 10^{#1}}}
\newcommand{\s}[1]{\setcounter{enumi}{#1}}

\begin{document}

\begin{flushright}{\large Homework 3\\ Mike Skalnik}\end{flushright}

Section 4.6

\begin{enumerate}
  \s{1}

  \item You will have to place, at most, $m n + 1$ pigeons into 1 hole.

  \item No, this simply not possible. Any 2-element subset of X will also be included in the 3-element subsets of X. This is the same if X was a set of positive integers from 1 to 6.

\end{enumerate}

Section 5.8

\begin{enumerate}

  \item
    \begin{enumerate}
      \setcounter{enumii}{4}

       \item 2

       \item 3

    \end{enumerate}

  \s{2}

  \item Such a graph cannot exist, since $\sum_{v \in V} \mathrm{deg}(v) > 2 |E|$.

  \s{5}

  \item For a tree with a single vertex, there are no edges. We then assume we have a tree with n-1 vertices that contains n-2 edges. We can construct a new tree with n vertices and n-1 edges by adding a leaf to that tree.

  \item We can quickly determine that $G_4$ is not isomorphic with any other graph, since the degree of vertex $x_5$ is 5, which isn't seen in any other graph. $G_2$ is also not isomorphic with any other graph since both $G_3$ and $G_1$ have a triangle that cannot be replicated with $G_2$. This leaves $G_1$ and $G_3$ which are isomorphic, the isomorphism being:
    \begin{eqnarray*}
      f\left(v_1\right) &=& w_1 \\ 
      f\left(v_2\right) &=& w_3 \\
      f\left(v_3\right) &=& w_4 \\
      f\left(v_4\right) &=& w_2 \\
      f\left(v_5\right) &=& w_6 \\
      f\left(v_6\right) &=& w_5
    \end{eqnarray*}

  \s{8}

  \item The given graph is eulerian since every vertex has an even degree. The eulerian curcuit is:
    \begin{eqnarray*}
      C &=& \left(a\right) \\
        & & \left(a, b, l, a\right) \\
        & & \left(a, b, l, d, j, l, a\right) \\
        & & \left(a, b, l, d, h, m, i, d, j, l, a\right) \\
        & & \left(a, b, l, d, h, m, g, n, m, i, d, j, l, a\right) \\
        & & \left(a, b, l, d, h, m, g, n, m, i, c, f, i, d, j, c, f, e, a, k\right)
    \end{eqnarray*}

  \s{12}
  
  \item The chromatic number is 3, and the graph coloring is: \\
    \begin{tabular}{l c r}
      a &:& 2 \\
      b &:& 2 \\
      c &:& 1 \\
      d &:& 1 \\
      e &:& 3 \\
      f &:& 2 \\
      g &:& 1 \\
      h &:& 1 \\
      i &:& 2 \\
      j &:& 2
    \end{tabular}
\end{enumerate}
\end{document}

